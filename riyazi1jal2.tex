\documentclass[12pt,a4paper]{article}
\usepackage{amsmath}
\usepackage{amsfonts}
\usepackage{amssymb}
\usepackage{amsthm}
\usepackage{framed}
\usepackage{enumerate}
\theoremstyle{definition}
\newtheorem{thm}{قضیه}
\newtheorem{mesal}[thm]{مثال}
\newtheorem{soal}[thm]{سوال}
\newtheorem{tav}[thm]{توجه}

\newtheorem{tamrinetahvili}{تمرین تحویلی}
\newtheorem{lem}[thm]{لم}

\newtheorem{nokte}[thm]{نکته}
\newtheorem{jambandi}{جمعبندی}
\newtheorem{defn}[thm]{تعریف}

\usepackage{xepersian}
\settextfont{XB Niloofar}
\setdigitfont{XB Niloofar}
\linespread{1.5}
\begin{document}
\section*{جلسه‌ی دوم}
ادامه‌ی مثالها:
\begin{mesal}
دنباله‌ی 
$a_n=(r^n)_{n\in \mathbb{N}}$
را در نظر بگیرید که در آن 
$r$
یک عدد گویای ثابت است و 
$0<r<1$.
نشان دهید که 
$\lim_{n \to \infty}a_n=0$.
\end{mesal}
اگر فرض کنیم
$r=\frac{1}{2}$
چند جمله‌ی اول دنباله به صورت زیرند
\[
\frac{1}{2},\frac{1}{2^2},\frac{1}{2^3},...
\]
بنابراین این ادعا که دنباله‌ی یادشده به صفر می‌گراید درست به نظر می‌رسد.
\begin{proof}[پاسخ]
باید ثابت کنیم که 
\[
\forall \epsilon >0 \quad \exists N_\epsilon \in \mathbb{N} \quad \forall n>N_\epsilon \quad r^n < \epsilon
\]
علت این که ننوشته‌ایم
$|r^n|<\epsilon$
این است که می‌دانیم جملات این دنباله همه مثبتند.
فرض کنید 
$\epsilon >0$
داده شده باشد.
توجه کنید که 
$r^n<\epsilon$
معادل است با 
$\frac{1}{r^n}> \frac{1}{\epsilon}$.
طبق فرض سوال می‌دانیم که
$0<r<1$
پس
$\frac{1}{r}>1$.
بنابراین می‌توانیم فرض کنیم که یک عددِ
$a>0$
موجود است به طوری که
$\frac{1}{r}=1+a$.
پس می‌خواهیم که
\[
 \frac{1}{r^n}={(\frac{1}{r})}^n={(1+a)}^n > \frac{1}{\epsilon}
\]
بنا به نامساوی برنولی
$(1+a)^n \geqslant 1+na$.
پس
کافی است داشته باشیم: 
\[
1+na>\frac{1}{\epsilon} 
\]
و برای آن کافی است که
\[
n>\frac{\frac{1}{\epsilon}-1}{a}.
\]
پس اگر
\[
N_\epsilon = \lfloor \frac{\frac{1}{\epsilon}-1}{a} \rfloor +1
\]
آنگاه
\[
\forall n>N_\epsilon \quad r^n <\epsilon
\]
از جمله‌ی 
$a_{N_\epsilon}$
به بعدِ دنباله مد نظر مااست. یعنی اگر
$a_n$
یکی از اعضای مجموعه‌ی زیر باشد 
\[
a_{N_\epsilon},a_{N_\epsilon+1},a_{N_\epsilon+2},\ldots 
\]
آنگاه 
$|a_n|<\epsilon$.
\end{proof}
\begin{thm}\hfill 
\begin{enumerate}[a.]
\item
فرض کنید 
$(a_n)_ {n \in \mathbb{N}}$
و 
$(b_n)_{n \in \mathbb{N}} $
دو دنباله‌ی همگرا باشند، آنگاه
\[
\lim_{n \to \infty}(a_n+b_n)=\lim_{n \to \infty}a_n+\lim_{n \to \infty}b_n
\]
\begin{proof}
فرض کنیم
که
\[
\lim_{n\to \infty} a_n=A, \quad \lim_{n\to \infty}b_n=B.
\]
برای این که نشان دهیم که
$\lim a_n+b_b=A+B$
باید نشان دهیم که
\[
\forall \epsilon >0 \quad \exists N_\epsilon \in \mathbb{N} \quad \forall n>N_\epsilon \quad |(a_n+b_n)-(A+B)| < \epsilon
\]
فرض کنیم
$\epsilon>0$
داده شده باشد. 
از آنجا که
$a_n$
همگرا به 
$A$
است
می‌دانیم که یک
$N^1_{\epsilon/2}$
موجود است، به طوری که
\[
\forall n>N^1_{\epsilon/2} \quad |a_n-A|<\epsilon/2
\]
همچنین
از آنجا که
$b_n$
همگرا به 
$B$
است
می‌دانیم که یک
$N^1_{\epsilon/2}$
موجود است، به طوری که
\[
\forall n>N^1_{\epsilon/2} \quad |b_n-B|<\epsilon/2
\]
پس اگر
$N>\max\{N^1_{\epsilon/1},N^1_{\epsilon/2}\}$
آنگاه 
\[
\forall n>N \quad |(a_n+b_n)-(A+B)|\leq |a_n-A|+|b_n-B|\leq \frac{\epsilon}{2}+\frac{\epsilon}{2}=\epsilon.
\]
\end{proof}
\item
\[
\forall \lambda \in \mathbb{R} \quad \lim_{n \to \infty}\lambda a_n = \lambda \lim_{n \to \infty}a_n
\]
\begin{proof}
فرض کنیم که
\[
\lim_{n\to \infty} a_n=A.
\]
باید نشان دهیم
که
\[
\forall\epsilon>0 \exists N\in \mathbb{N}\quad \forall n>N
\quad |\lambda a_n-\lambda A|=|\lambda| |(a_n-A)|<\epsilon.
\]
کافی است بگیریم
$\epsilon_1=\frac{\epsilon}{|\lambda|}$
و از همگرائیِ دنباله‌ی
$a_n$
استفاده کنیم.
\end{proof}
\item
اگر 
$\lim_{n \to \infty}b_n \neq 0$
آنگاه
\[
\lim_{n \to \infty}\frac{a_n}{b_n}=\frac{\lim_{n \to \infty}a_n}{\lim_{n \to \infty}b_n}
\]
\begin{proof}
فرض کنیم که
$\lim_{n\to \infty}b_n=B\not=0$
و
$\lim_{n\to \infty}a_n=A$.
باید ثابت کنیم که
\[
\forall \epsilon>0 \quad \exists N\in \mathbb{N}\quad \forall n>N \quad 
|\frac{a_n}{b_n}-\frac{A}{B}|<\epsilon
\]
پس می‌خواهیم که داشته باشیم
\[
|\frac{a_nB-Ab_n}{Bb_n}|<\epsilon
\]
عبارتِ
$-AB+AB$
را به درون صورت اضافه می‌کنیم:
\[
|\frac{a_nB-AB+AB-Ab_n}{Bb_n}|
<\epsilon
\]
داریم
\[
|\frac{a_nB-AB+AB-Ab_n}{Bb_n}|\leq \frac{|B||a_n-A|+|A||b_n-B|}{|{Bb_n}|}
\]
کافی است عبارت سمت راستِ بالا از
$\epsilon$
کمتر باشد.
توجه کنید که از آنجا که
$b_n$
همگراست، یک
$N_1$
موجود است به طوری که 
\[
\forall n>N_1 \quad |b_n-B|<\epsilon \quad (*)
\]
پس
\[
\forall n>N_1 \quad B-\epsilon<b_n<B+\epsilon\quad (**)
\]
بنا به 
$(**)$
می‌توان اعداد مثبتِ
$M_1,M_2$
را چنان یافت که
\[
\forall n\in \mathbb{N}\quad  M_1<|b_n|<M_2 \quad (***).
\]
حال توجه کنید که دنباله‌ی
$a_n$
به 
$A$ 
همگراست. پس عددِطبیعی
$N_2$
چنان موجود است که
\[
\forall n>N_2 \quad |a_n-A|<\epsilon.
\]
حال اگر
$n>\max\{N_1,N_2\}$
آنگاه
\[
|a_n-A|<\epsilon, \quad |b_n-B|<\epsilon
\]
پس
\[
\frac{|B||a_n-A|+|A||b_n-B|}{|{Bb_n}|}\leq
\frac{|B|\epsilon+|A|\epsilon}{|B|M_1}=\frac{(|A|+|B|)\epsilon}{|B|M_1}
\]
بحث تقریباًتمام شده است؛
تا اینجا ثابت کرده‌ایم که:
\newline 
 برای هر
$\epsilon>0$
عددِ
$N\in \mathbb{N}$
چنان موجود است که 
\[
\forall n>N \quad |\frac{a_n}{b_n}-\frac{A}{B}|<\frac{(|A|+|B|)\epsilon}{|B|M_1}
\]
در بند بالا، به جای
$\epsilon$
مقدارِ
$\frac{|B|M_1}{|A|+|B|}\epsilon$
را بگذارید.
\footnote{نه! در امتحان نمی‌آید!}
\end{proof}
\end{enumerate}
\end{thm}
\begin{mesal}
حد دنباله‌های زیر را بیابید.
\begin{itemize}
\item
\[
a_n=\lim_{n \to \infty}\frac{4^{n+3}+7}{5^n}
\]
\begin{proof}[پاسخ]
\[
\lim_{n \to \infty}a_n = \lim_{n \to \infty}\frac{4^{n+3}}{5^n}+\lim_{n \to \infty}\frac{7}{5^n}
\]
\[
\lim_{n \to \infty}a_n = \lim_{n \to \infty}(\frac{4}{5})^n\times 4^3+\lim_{n \to \infty}\frac{1}{5^n}\times 7=0+0=0
\]
\end{proof}
\item
\[
a_n = \frac{3^n}{2^n+4^n}
\]
راهنمایی: صورت و مخرج را بر
$4^n$
تقسیم کنید.
\end{itemize}
\end{mesal}
در قضیه‌ی زیر می‌بینیم که اگر دنباله‌ای میان دو دنباله‌ی همگرا فشرده شود، همگراست. فرض کنیم
$\lim a_n=L, \lim b_n=L$
و
$a_n\leq c_n\leq b_n$.
برای 
$n$
های به‌اندازه‌ی کافی بزرگ جملات دنباله‌های
$a_n,b_n$
به 
$L$
نزدیکند. جملات دنباله‌ی
$c_n$
که میان این دو دنباله هستند نیز به ناچار در نزدیکی 
$L$
قرار می‌گیرند. در زیر این گفته را دقیق بیان و اثبات کرده‌ایم.
\LTRfootnote{Squeeze Lemma}
\begin{thm}[فشردگی]
اگر 
$\lim_{n\to \infty} a_n = \lim_{n\to \infty} b_n = L$
و 
\[
\forall n \in \mathbb{N} \quad a_n\leqslant c_n\leqslant b_n,
\]
آنگاه
\[
\lim_{n\to \infty} c_n =L
\]
\begin{proof}
باید ثابت کنیم که
\[
\forall \epsilon >0 \quad \exists N_\epsilon\in \mathbb{N} \quad \forall n>N_\epsilon \quad |c_n-L|<\epsilon
\]
یعنی می‌خواهیم
\[
\forall \epsilon >0 \quad \exists N_\epsilon\in \mathbb{N} \quad \forall n>N_\epsilon \quad L-\epsilon<c_n<L+\epsilon
\]
فرض کنیم که
$\epsilon>0$
داده شده باشد.
از آنجا که 
$\lim_{n\to \infty} a_n=L$
 می‌دانیم که 
$N_1\in \mathbb{N}$
چنان موجود است که 
\[
\forall n>N_1 \quad a_n<L+\epsilon
\]
نیز از آنجا که 
$\lim b_n=L$
می‌دانیم که 
$N_2\in \mathbb{N}$
چنان موجود است که 
\[
\forall n>N_2 \quad L-\epsilon<b_n
\]
پس اگر
$n>\max\{\mathbb{N}_1,\mathbb{N}_2\}$
آنگاه
\[
L_\epsilon<b_n\leq c_n\leq a_n<L+\epsilon.
\]
\end{proof}
\end{thm}
\begin{mesal}
با استفاده از قضیه‌ی فشردگی
ثابت کنید که 
\[
\lim_{n \to \infty}\frac{2^n}{n!}=0
\]
\end{mesal}
\begin{proof}[پاسخ]
داریم
\[
0 \leqslant
\frac{2^n}{n!}=
 \frac{\overbrace{2 \times 2 \times 2 \times ... \times 2}^{\text{ بار}n}}{1 \times 2 \times \underbrace{ \times ... \times n}_{\geq 3 \times ... \times 3}} \leqslant 2 \times \frac{2^{n-2}}{3^{n-2}}=2 \times (\frac{2}{3})^{n-2}
\]
دنباله‌ی ثابتِ
$0$
و دنباله‌ی
$2 \times (\frac{2}{3})^{n-2}$
هر دو به صفر میل می‌کنند، پس
بنا به فشردگی
\[
\lim \frac{2^n}{n!}=0.
\]
\end{proof}
\begin{tav}
همان اثبات بالا نشان می‌دهد که برای هر
$a>0$
داریم
$\lim _{n\to \infty}\frac{a^n}{n!}=0$.
\end{tav}
\begin{tav}
از آنجا که
$\lim _{n\to \infty}\frac{2^n}{n!}=0$
برای هر
$\epsilon>0$
دلخواه، یک
$N\in\mathbb{N}$
چنان یافت می‌شود که 
\[
\forall n>N \quad \frac{2^n}{n!}<\epsilon
\]
یعنی
\[
\forall n>N \quad 2^n<\epsilon n!
\]
و این  تقریباً همان «نرخ‌ رشد» است که درباره‌اش صحبت کرده‌ایم. 
\end{tav}
\begin{mesal}
قرار دهید 
$a_n = \sqrt[n]{n}$
و نشان دهید که 
\[
\lim_{n \to \infty}a_n=1
\]
\end{mesal}
\begin{proof}[پاسخ]
چند جمله‌ی اول دنباله به صورت زیرند:
\[
1 \quad \sqrt{2} \quad \sqrt[3]{3} \quad \sqrt[4]{4} \quad ...
\]
داریم
\[
a_n = \sqrt[n]{n} \geqslant \sqrt[n]{1} =1
\]
پس می‌توان نوشت
\[
a_n=1+b_n \quad b_n \geqslant 0
\]
نشان می‌دهیم که 
$\lim_{n \to \infty}b_n=0$
\[
a_n=\sqrt[n]{n}=1+b_n \quad \Rightarrow \quad n=(1+b_n)^n = 1+nb_n+\binom{n}{2}b_n^2+... 
\]
\[
\Rightarrow \quad n\geqslant \binom{n}{2}b_n^2=\frac{n(n-1)}{2}b_n^2
\]
\[
\Rightarrow \quad b_n^2 \leqslant \frac{2}{n-1} \Rightarrow 0 \leqslant b_n \leqslant \sqrt{\frac{2}{n-1}}
\]
بنا به فشردگی
\[
\lim_{n \to \infty}b_n=0
\]
\end{proof}
\begin{mesal}
اگر
$a_n=\sqrt[n]{1+2^n}$
نشان دهید که 
\[
\lim_{n \to \infty}a_n=2
\]
\end{mesal}
\begin{proof}[پاسخ]
\[
a_n=\sqrt[n]{1+2^n} \geqslant \sqrt[n]{2^n}=2 \quad \Rightarrow \quad a_n\geqslant 2 \quad \Rightarrow \quad \frac{a_n}{2}\geqslant 1
\]
\[
(a_n)^n = 1+2^n \quad \Rightarrow \quad (\frac{a_n}{2})^2 = \frac{1}{2^n}+1
\]
\[
1 \leqslant \frac{a_n}{2} \leqslant \underbrace{(\frac{a_n}{2})^n}_{\lim_{n \to \infty}(\frac{a_n}{2})^n = 1}
\]
\[
\lim_{n \to \infty}\frac{a_n}{2} = 1 \quad \Rightarrow \quad \lim_{n \to \infty}a_n = 2
\]
\end{proof}
\begin{defn}[دنباله‌ی کراندار]
دنباله‌ی
$(a_n)$
را کراندار می‌خوانیم هرگاه
\[
\exists M \in \mathbb{N} \quad \forall n \in \mathbb{N} \quad |a_n|<M
\]
یعنی
\[
\forall n \in \mathbb{N}\quad 
-M<a_n<M.
\]
\end{defn}
\begin{proof}[مشاهده]
هر دنباله‌ی همگرا کراندار است.
\[
a_n \mapsto L
\]
\[
\epsilon =\frac{1}{2} \quad \exists N_\epsilon \quad \forall n>N_\epsilon \quad |a_n-L|<\frac{1}{2}
\]
\[
\Rightarrow \quad \forall n > N_\epsilon \quad L-\frac{1}{2} <a_n<L+\frac{1}{2}
\]
\end{proof}
\begin{tav}
$(-1)^n$
کراندار نیست ولی همگراست.
\end{tav}
\begin{thm}
هر دنباله‌ی صعودی و از بالا کراندار همگراست (و هر دنباله‌ی نزولی و از پائین‌کراندار همگراست).
\end{thm}
یک دنباله‌ی صعودی و از بالاکراندار به کوچکترین کرانِ بالای خود همگراست. 
وجودِ کوچکترین کرانِ بالا را اصل تمامیت در اعدادِ حقیقی تضمین می‌کند:
\begin{tav}
هر زیرمجموعه‌ی کراندار از اعدادِ حقیقی، دارای کوچکترین کران بالا است.
\end{tav}
آیا آنچه در بالا گفته‌ایم درباره‌ی اعدادِ گویا هم درست است؟
\begin{mesal}
نشان دهید که دنباله‌ی 
$a_n=\sum_{k=1}^n \frac{1}{k!}$
همگراست.
\end{mesal}
\begin{proof}[پاسخ]
چند جمله‌ی اول دنباله به صورت زیرند:
\[
a_1=1 \quad a_2=1+\frac{1}{2!} \quad a_3=1+\frac{1}{2!}+\frac{1}{3!}
\]
دقت کنید که
\[
a_n-a_{n-1}=\frac{1}{n!} \geqslant 0
\]
یعنی دنباله‌ی
$(a_n)$
صعودی است. کافی است نشان دهیم که دنباله‌ی یادشده از بالا کراندار است. 
\[
a_n=\sum_{k=1}^n\frac{1}{k!}=1+\frac{1}{2!}+\frac{1}{3!}+...+\frac{1}{n!} \leqslant 1+\frac{1}{2}+\frac{1}{2\times 2}+\frac{1}{2\times 2\times 2}+...+\frac{1}{2^{n-1}}
\]
\[
a_n \leqslant \underbrace{(\frac{1}{2})^0+(\frac{1}{2})^1+(\frac{1}{2})^2+...+(\frac{1}{2})^{n-1}}_{= \frac{1-(\frac{1}{2})^n}{1-\frac{1}{2}}=\frac{1-(\frac{1}{2})^n}{\frac{1}{2}}=2(1-(\frac{1}{2})^n) \leqslant 3}
\]
\end{proof}
در جلسات بعد این را که
\[
\frac{1}{2})^1+(\frac{1}{2})^2+...+(\frac{1}{2})^{n-1}=\frac{1-(\frac{1}{2})^n}{1-\frac{1}{2}}
\]
ثابت خواهیم کرد.
\begin{framed}
در این جلسه نشان دادیم که
\begin{itemize}
\item 
برای هر عددِ حقیقیِ
$0<r<1$
داریم
$\lim_{n\to \infty} r^n=0$.
\item 
$ \lim_{n\to \infty}\sqrt[n]{n}=1$.
\item 
دنباله‌ی 
$a_n=\sum_{k=1}^n \frac{1}{k!}$
همگراست.
\end{itemize}

\end{framed}
\end{document}