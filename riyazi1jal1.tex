\documentclass[12pt,a4paper]{article}
\usepackage{amsmath}
\usepackage{amsfonts}
\usepackage{amssymb}
\usepackage{amsthm}
\usepackage{framed}
\usepackage{hyperref}


\theoremstyle{definition}
\newtheorem{thm}{قضیه}
\newtheorem{mesal}[thm]{مثال}
\newtheorem{soal}[thm]{سوال}
\newtheorem{tav}[thm]{توجه}

\newtheorem{defn}[thm]{تعریف}

\usepackage{xepersian}
\settextfont{XB Niloofar}
\setdigitfont{XB Niloofar}
\linespread{1.5}
\begin{document}
\section{جلسه‌ی اول}
پیش از آنکه درس را رسماً شروع کنیم درباره‌ی حساب توضیح کوتاهی می‌دهم.
واژه‌ی 
Calculus
در لاتین به معنی سنگ کوچکی است که در چرتکه‌های دستی از آن استفاده می‌شود. 
این واژه را، در لغت اصطلاحی آن، حسابان (یا حساب) ترجمه کرده‌اند.
\par 
حسابان اشاره به دو حساب دارد، حساب دیفرانسیل و حساب انتگرال.
آنجا که صحبت از تغییرات یک کمیت بر حسب تغییرات بی‌نهایت کوچک کمیت دیگری است، با 
\textbf{حساب 
دیفرانسیل 
}
سر و کار داریم. مثلاً 
سرعت متوسط یک جسم در بین زمانهای 
$t$
و
$t+\Delta t$
برابر است با 
$\frac{\Delta x}{\Delta t}$
که در آن
$\Delta x$
میزان جابجایی جسم است. حال
برای محاسبه‌ی سرعت لحظه‌ای یک جسم در لحظه‌ی 
$t$
 باید میزان
تغییر مکان آن را 
در زمانی بی‌نهایت کوچک پس از
$t$
بدانیم. 
این کمیت را با
$\frac{dx}{dt}$
نشان می‌دهیم. 
مفهوم بی‌نهایت کوچک از 
مفاهیم مشکل‌ساز است.
\par 
در حساب بی‌نهایت کوچک را با نزدیک شدن به اندازه‌ی کافی تعبیر می‌کنند. مثلا منظور از این که
سرعت جسم در لحظه‌ی
$t$
برابر است با 
$v$
این است که 
\[
\lim_{\Delta t\to 0} \Delta x/\Delta t =v.
\]
یعنی می‌شود زمانها را «به اندازه‌ی کافی» کوچکتر و کوچکتر کرد و بدینسان به  «اندازه‌ی دلخواه» به سرعتِ
لحظه‌ایِ
$v$
نزدیک شد. 
\par 
گفتم که درک بی‌نهایت نزدیک شدن به چیزی دشوار است. مؤید این گفته، تناقض خرگوش و لاک‌پشت است. 
فرض کنید خرگوشی با لاک‌پشتی وارد مسابقه سرعت شده است. سرعت خرگوش چندین برابر از سرعت لاک پشت بیشتر است، اما خرگوش ده قدم عقبتر از لاک‌پشت ایستاده است. 
آنها همزمان شروع به دویدن می‌کنند. با استدلال زیر، خرگوش هیچگاه به لاک‌پشت نمی‌رسد:
برای این که خرگوش به لاک‌پشت برسد، باید نخست به نقطه‌ای برسد که لاک‌پشت
در آن است. تا زمانی که خرگوش بدان نقطه برسد لاک‌پشت از آن نقطه رفته است!
\par 
بخش دیگر حساب،\textbf{ حساب انتگرال }است: برای محاسبه‌ی مساحت زیر یک منحنی، به 
«تعدادی کافی»
مستطیل زیر آن نیازمندیم که 
مجموع مساحت آنها «به هر اندازه‌ی دلخواه» به مساحت زیر منحنی نزدیک شود.
ارتفاع این مستطیلها برابر با
$f(x)$
و قاعده‌ی آنها برابر با
$dx$
است. جمع این مقادیر، یعنی عبارت
$\sum f(x).dx$
را با
$\int f(x)dx$
نشان می‌دهیم. 
\par 
حساب دیفرانسیل و انتگرال در واقع یک حسابند! بنا به قضیه‌ی اساسی حساب:
\[
\int_a^b f'(x)dx=f(b)-f(a)
\]
یعنی انتگرالِ مشتق می‌شود خودِ‌تابع.
\section*{چند رابطه‌ی مهم}
برای ورود به بحث نیازمند یادآوری روابط زیر هستیم:
\begin{itemize}
\item
نامساوی برنولی:
\[
\forall a\geqslant -1 \in \mathbb{R}\quad \forall n\in \mathbb{N}\quad  (1+a)^n \geqslant 1+na
\]
\item
\[
1+2+...+n=\frac{n(n+1)}{2}
\]
\item
\[1^2+2^2+...+n^2=\frac{n(n+1)(2n+1)}{6}\]
\item
\[a^0+a^1+a^2+...+a^n=\frac{1-a^{n+1}}{1-a}\]
\item
\[(a+b)^n=\sum_{i=0}^n \binom{n}{i} a^i b^{n-i} \]
که در آن
$ \binom{n}{i}= \sum \frac{n! }{i! (n-i)!}$.
\end{itemize}
\section*{دنباله‌ها}
دنباله برای ما یعنی لیستی نامتناهی از اعداد حقیقی به صورت زیر:
\[
a_1,a_2,\ldots
\]
هر لیست نامتناهی توسط اعداد طبیعی شمرده می‌شود. پس بیایید دنباله‌ها را دقیقتر تعریف کنیم.
\begin{defn}
دنباله یعنی تابعی از 
$\mathbb{N}$
به
$\mathbb{R}$
به صورت زیر
\begin{align*}
&
f: \mathbb{N} \to \mathbb{R}, 
\\
&
n\mapsto a_n
\end{align*}
\end{defn}
هرگاه ضابطه‌ی 
$f$
معلوم باشد، 
$a_n$
 را جمله‌ی عمومی دنباله می‌خوانیم.
\begin{tav}
دنباله را با نمادهای 
$\{ a_n \}^\infty_{n=1}$،
$(a_n)^\infty_{n=1}$،
$\{ a_n \}$و
$(a_n)$
نشان می‌دهیم.
\end{tav}

\begin{mesal}
جمله‌ی عمومی دنباله‌ی زیر را بیابید.
\[\frac{3}{5},\frac{-4}{5},\frac{5}{125},\frac{-6}{625},\frac{7}{3125},...\]
\end{mesal}
\begin{proof}[پاسخ]
$a_n=\frac{(-1)^{n+1}n+2}{5^n}$
\end{proof}
\begin{mesal}
چند جمله‌ی اول دنباله‌ی زیر را بنویسید.
\[f:\mathbb{N} \to \mathbb{R},\quad 
 f(n)=\sum^n_{k=1} \frac{1}{k!}\]
\end{mesal}
\begin{proof}[پاسخ]
حل:
\begin{itemize}
\item
$a_1=\frac{1}{1!}$
\item
$a_2=\frac{1}{1!}+\frac{1}{2!}$
\item
$a_3=\frac{1}{1!}+\frac{1}{2!}++\frac{1}{3!}$
\end{itemize}
\end{proof}
لزوما دنباله‌ها دارای جمله‌ی عمومی مشخص نیستند: فرض کنید
$a_n$
جمعیت جهان باشد در اول مهرماهِ
$n$
سال پس از امسال. یا فرض کنید
$b_n$
برابر باشد با
$n$
امین رقم بعد از اعشار در بسط 
اعشاری
عددِ
$\pi$.
\par 
گاهی ضابطه‌ی یک دنباله به صورت \textbf{بازگشتی} تعریف می‌شود.
فیبوناچی (در قرن ۱۳ میلادی) سوال زیر را پرسیده است:
فرض کنیم یک جفت خرگوش داریم و بدانیم که هر جفت خرگوش بعد از دو ماه، ماهی یک جفت دیگر تولید می‌کنند. تعداد خرگوش‌ها را در ماه 
$n$ام
 بیابید.
\begin{proof}[پاسخ]
\hfill 
\begin{itemize}
\item 
$a_1=1$
یعنی در ماه اول یک جفت خرگوش ۰ ماهه داریم.
\item 
$a_2=1$
در ماه دوم یک جفت خرگوش یک ماهه داریم.
\item 
در ماه سوم، یک جفت خرگوش دو ماهه داریم که یک جفت خرگوش ۰ ماهه تولید می‌کند، پس
$a_3=1+1=2=a_1+a_2$.
\item 
بدین‌ترتیب در ماه چهارم
یک جفت خرگوش ۳ ماهه داریم که یک جفت تازه تولید می‌کند و یک جفت خرگوش ۱ ماهه؛ پس
$a_4=1+1+1=3=a_3+a_2$.
\item 
و بدین صورت می‌توان بررسی کرد که
$a_{n+2}=a_{n}+a_{n+1}$.
\end{itemize}
دنباله‌ی فیبوناچی به خاطر خرگوشها فقط اهمیت ندارد!‌
پیشنهاد می‌کنم در صفحه‌ی ویکی‌پدیا درباره‌ی این دنباله بیشتر مطالعه کنید:
\url{https://en.wikipedia.org/wiki/Fibonacci_number}
\end{proof}
\begin{mesal}
جمله‌ی عمومی دنباله‌ی زیر را به صورت بازگشتی بنویسید:
\[
a_1=1, a_2=\sqrt{2}, a_3=\sqrt{1+\sqrt{2}}, a_4=\sqrt{1+\sqrt{1+\sqrt{2}}}, \ldots\]
\end{mesal}
\begin{proof}[پاسخ]
$a_1 = 1, a_{n+1}=\sqrt{1+a_n} $
\end{proof}
\section*{حد دنباله‌ها}
دنباله‌ی
$\frac{1}{n}$
را در نظر بگیرید. هر چند انتهای این دنباله معلوم نیست ولی به نظر می‌آید هر چه
$n$
بزرگتر می‌شود، جملات دنباله بیشتر در نزدیکی صفر تجمع می‌کنند. چگونه می‌توانیم بگوییم که جملات این دنباله بی‌نهایت به صفر نزدیک می‌شوند؟
\newline
\textbf{تعریف غیر رسمی:}
می‌گوییم دنباله‌ی
$a_n$
به 
$L$
همگراست هرگاه 
$a_n$ها
به هر اندازه‌ی دلخواه از 
یک
$n$
به اندازه‌ی کافی بزرگ
 (وابسته به اندازه‌ی دلخواه ما)
 به بعد به 
$L$
نزدیک شوند.
\newline
در تعریف بالا دو عبارت «اندازه‌ی دلخواه» و «اندازه‌ی کافی» نقش کلیدی بازی می‌کنند. 
از آنجا که «بی‌نهایت نزدیک شدن» را مستقیماً نمی‌توان نوشت،
برای بیان این که فاصله‌ی که جملات این دنباله
از حدشان بی‌نهایت کوچک است، به این دو تعبیر نیازمندیم. 
\begin{defn}[ریاضی]
\[
\lim_{n \to \infty} a_n =L \iff \forall \epsilon>0\quad \exists  N_\epsilon\in \mathbb{N} \quad \forall n > \mathbb{N} \quad |a_n-L|<\epsilon.
\]
\end{defn}
پس وقتی ادعا می‌کنید که حد دنباله‌ی
$a_n$
برابر با
$L$
است، باید برای هر 
$\epsilon$
که من به شما بدهم، شما یک
$N_\epsilon$
به من بازگردانید به طوری که مطمئن شوم که همه‌ی جملاتِ
\[
a_N,a_{N+1},a_{N+2},\ldots
\]
در بازه‌ی
$(L-\epsilon,L+\epsilon)$
واقع می‌شوند (يعنی به اندازه‌ی
$\epsilon$
به
$L$
نزدیکند).
\begin{defn}
دنباله‌ی
$a_n$
را \textbf{همگرا} می‌خوانیم هرگاه
\[\exists L \quad \lim_{n \to \infty} a_n = L.\]
در غیر این صورت، این دنباله را \textbf{واگرا} 
می‌خوانیم. 
\end{defn}
\begin{mesal}
ثابت کنید که 
$\lim_{n \to \infty} \frac{1}{n}=0$.
\newline
پاسخ:
فرض کنیم 
$\epsilon >0$
داده شده باشد و بخواهیم 
$N_\epsilon$
را طوری بیابیم که برای 
$n>N_\epsilon$
داشته باشیم
$|\frac{1}{n}|<\epsilon$.
برای اینکه 
$|\frac{1}{n}|<\epsilon$
باید داشته باشیم
$n>\frac{1}{\epsilon}$.
واضح است که برای هر برای هر عدد طبیعیِ
$n \frac{1}{\epsilon}$
داریم
$ |\frac{1}{n}|<\epsilon$.
پس قرار می‌دهیم
$N_\epsilon=[\frac{1}{\epsilon}]+1$.
داریم
\[
\forall \epsilon>0 \quad \forall n>N_\epsilon=[1/\epsilon]+1
\quad |1/n|<\epsilon.\]
\end{mesal}
\begin{mesal}
فرض کنید که
$r$
یک عدد گویای مثبت باشد.
ثابت کنید که 
$\lim_{n \to \infty} \frac{1}{n^r}=0$
\newline
پاسخ:
فرض کنیم 
$\epsilon >0$
داده شده باشد و بخواهیم 
$|\frac{1}{n^r}|<\epsilon$.
پس می‌خواهیم 
$n^r>\frac{1}{\epsilon}$
یعنی
$n>\sqrt[r]{\frac{1}{\epsilon}}$.
اگر
$N$
یک عدد طبیعی بزرگتر از 
$\sqrt[r]{\frac{1}{\epsilon}}$
باشد آنگاه
\[
\forall n>N \quad |\frac{1}{n^r}|<\epsilon.
\]
\end{mesal}
در مثال بالا 
$r$
را
$\frac{2}{3}$
بگیرید و حاصل را تحقیق کنید. 
\begin{mesal}
ثابت کنید که 
$\lim_{n \to \infty} \frac{4n^2+2n}{n^2+2}=4$
\newline
پاسخ:
باید نشان داد که
\[
\forall \epsilon>0 \quad \exists N_\epsilon\in \mathbb{N}
\quad \forall n>N_\epsilon \quad |\frac{4n^2+2n}{n^2+2}-4|<\epsilon.
\]
فرض کنیم 
$\epsilon >0$
داده شده باشد و بخواهیم که برای
$n$
های بزرگتر از
یک
$N_\epsilon$
داشته باشیم
$|\frac{4n^2+2n}{n^2+2}-4|<\epsilon$.
\begin{framed}
محاسبات: 
\begin{align*}
& |\frac{4n^2+2n}{n^2+2}-4|<\epsilon\Rightarrow |\frac{4n^2+2n-4n^2-8}{n^2+2}|<\epsilon\Rightarrow |\frac{2n-8}{n^2+2}|<\epsilon\Rightarrow\\
& |\frac{n^2+2}{2n-8}|<\frac{1}{\epsilon}.
\end{align*}
\end{framed}
پس می‌خواهیم از جایی به بعد داشته باشیم
\[
\frac{n^2+2}{2n-8}>\frac{1}{\epsilon}.
\]
توجه کنید که
$\frac{n^2+2}{2n-8}\geq \frac{n^2}{2n}=\frac{n}{2}$.
پس هر جا که
$\frac{n}{2}>\frac{1}{\epsilon}$
واضح است که 
$\frac{n^2+2}{2n-8}>\frac{1}{\epsilon}$.
اگر
$N>\frac{2}{\epsilon}$
یک عدد طبیعی باشد، آنگاه 
\[
\forall n>N \quad \frac{n}{2}>\frac{1}{\epsilon}
\]
پس
\[
\forall n>N \quad \frac{n^2+2}{2n-8}>\frac{1}{\epsilon}
\]
پس
\[
\forall n>N \quad \frac{2n-8}{n^2+2}<\epsilon
\]
یعنی
\[
\forall n>N \quad |a_n-4|<\epsilon.
\]
\end{mesal}
\end{document}