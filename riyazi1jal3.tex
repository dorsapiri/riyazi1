\documentclass[12pt,a4paper]{article}

\usepackage{amsmath}
\usepackage{cancel} 

\usepackage{amsfonts}
\usepackage{amssymb}
\usepackage{amsthm}
\usepackage{framed}



\theoremstyle{definition}
\newtheorem{thm}{قضیه}
\newtheorem{mesal}[thm]{مثال}
\newtheorem{soal}[thm]{سوال}
\newtheorem{tav}[thm]{توجه}

\newtheorem{tamrinetahvili}{تمرین تحویلی}
\newtheorem{lem}[thm]{لم}

\newtheorem{nokte}[thm]{نکته}
\newtheorem{jambandi}{جمعبندی}
\newtheorem{defn}[thm]{تعریف}

\usepackage{xepersian}
\settextfont{XB Niloofar}
\setdigitfont{XB Niloofar}
\linespread{1.5}
\begin{document}
\section{نیم‌جلسه‌ی سوم}
\begin{mesal}
نشان دهید دنباله‌ی 
$(1+\frac{1}{n})^n$
همگراست.
\end{mesal}
\begin{proof}[پاسخ]
نشان می‌دهیم که دنباله‌ی یاد شده‌ی صعودی و از بالا کراندار است.
صعودی بودن دنباله یعنی:
\[
\forall n \quad a_n \leqslant a_{n+1}
\]
پس  از آنجا که جملات دنباله مثبتند،‌ کافی است برای اثبات صعودی بودن دنباله، نشان دهیم:
\[
\forall n \quad \frac{a_{n+1}}{a_n} \geqslant 1
\]
داریم
\[
\frac{a_{n+1}}{a_n}=
\frac{(1+\frac{1}{n+1})^{n+1}}{(1+\frac{1}{n})^n}=\frac{(\frac{n+2}{n+1})^{n+1}}{(\frac{n+1}{n})^n}\times \frac{\frac{n+1}{n}}{\frac{n+1}{n}}=(\frac{n+1}{n})(\frac{n(n+2)}{(n+1)^2})^{n+1}=(\frac{n+1}{n})(\frac{n^2+2n}{(n+1)^2})^{n+1}
\]
\[
=(\frac{n+1}{n})(\frac{(n+1)^2-1}{(n+1)^2})^{n+1}=(\frac{n+1}{n})(1-\frac{1}{(n+1)^2})^{n+1}
\]
با توجه به نامساوی برنولی، 
داریم
$(1+a)^{n+1} \geqslant 1+(n+1)a$
پس
\[
(\frac{n+1}{n})(1-\frac{1}{(n+1)^2})^{n+1} \geqslant \frac{n+1}{n} \times (1-\frac{1}{(n+1)^2}) =\frac{n+1}{n}-\frac{\cancel{n+1}}{n \cancel{(n+1)}}=1+\frac{1}{n}-\frac{1}{n}=1 
\]

\hfill $\square$
پایان اثبات صعودی بودن.
\newline 
اثبات کراندار بودن دنباله:
\[
a_n=(1+\frac{1}{n})^n=\sum_{k=0}^n \underbrace{\binom{n}{k} (\frac{1}{n})^k}_{\leqslant \frac{1}{k!}\text{ادعا:}}
\]
\[
\binom{n}{k}=\frac{n!}{k! (n-k)!}
\]
\[
\binom{n}{k} (\frac{1}{n})=\frac{1}{k!} \times \frac{n \times (n-1) \times ... \times (n-k+1)}{n^k}
\]
\[
\frac{n \times (n-1) \times ... \times (n-k+1)}{n^k} \leqslant 1
\]
\[
\Rightarrow \quad a_n \leqslant \sum_{k=0}^{n} \frac{1}{k!}
\]
جلسه‌ی قبل نشان دادیم که 
$\sum_{k=0}^{n} \frac{1}{k!}$
کراندار است.
\end{proof}
\begin{tav}
بعداً در همین درس خواهیم دید که حد دنباله‌ی
$a_n=(1+\frac{1}{n})^n$
برابر است با
$e$،
عددِ نپر. عددِ نپر همچنین برابر است با حاصلجمعِ سری زیر:
\[
\sum_{n=0}^\infty \frac{1}{n!}.
\]
\end{tav}
\begin{mesal}
حد دنباله‌ی زیر را با ذکر دلیل مشخص کنید.
\[
a_n=\sqrt{2n^5-5n}-\sqrt{2n^5-n^2}
\]
\end{mesal}
\begin{proof}[پاسخ]
\[
a_n=\underbrace{\sqrt{2n^5-5n}}_{a}-\underbrace{\sqrt{2n^5-n^2}}_b
\]
با توجه به رابطه‌ی 
$(a-b)(a+b)=a^2-b^2$
داریم:
\[
a_n=a_n \times \frac{a+b}{a+b}=\frac{\overbrace{5n}^{\leqslant 5n^2}+n^2}{\sqrt{2n^5-5n}+\sqrt{2n^5-n^2}} \geqslant 0
\]
مخرج کسر را کوچک و صورت آن را بزرگ می‌کنیم
\[
0 \leqslant a_n \leqslant \frac{6n^2}{\sqrt{2n^5}+\sqrt{n^5}}=\frac{6n^2}{(\sqrt{2}+1) n^{\frac{5}{2}}}=\frac{6}{(\sqrt{2}+1)}n^{2-\frac{5}{2}}
\]
$6n^{2-\frac{5}{2}}$
به صفر میل می‌کند. در نتیجه حد 
$a_n$
 نیز 
 بنا به فشردگی 
 صفر است. 
\end{proof}
\begin{mesal}
نشان دهید که 
$\lim_{n \to \infty}\sqrt[n]{2}=1$.
\end{mesal}
\begin{proof}[پاسخ]
\[
1=\sqrt[n]{1} \leqslant \sqrt[n]{2}=1+b_n \quad b_n \geqslant 0
\]
نشان می‌دهیم که 
$\lim_{n \to \infty} b_n =0$
می‌دانیم 
$b_n \geqslant 0$
همچنین
$2=(1+b_n)^n$
پس
\[
2=1+\binom{n}{1}b_n+\binom{n}{2}b_n^2+...+\binom{n}{n}b_n^n
\]
پس
\[
2\geq \binom{n}{1}b_n\Rightarrow
1 \geqslant nb_n
\]
\[
b_n \leqslant \frac{1}{n}
\]
بنابراین
\[
0 \leqslant b_n \leqslant \frac{1}{n}
\]
\[
\lim_{n \to \infty} \frac{1}{n}=0
\]
در نتیجه بنا به فشردگی حد دنباله‌ی 
$b_n$
نیز صفر است.
\end{proof}
\begin{tav}
به طور کاملاً مشابه می‌توان نشان داد که اگر 
$a>1$
آنگاه
\[
\sqrt[n]{a} \mapsto 1
\]
\end{tav}
\begin{mesal}
نشان دهید که 
$\lim_{n \to \infty}\sqrt[n]{2^n+3^n}=3$
\end{mesal}
\begin{proof}[پاسخ]
\[
\sqrt[n]{3^n} \leqslant \sqrt[n]{2^n+3^n} \leqslant \sqrt[n]{3^n+3^n}
\]
\[
\Rightarrow \quad 3 \leqslant a_n \leqslant \sqrt[n]{2 \times 3^n}
\]
\[
\Rightarrow \quad 3 \leqslant a_n \leqslant 3\sqrt[n]{2}
\]
در مثال قبل دیدیم 
که
$\sqrt[n]{2}$
به یک میل می‌کند، پس بنا به فشردگی
$a_n \mapsto 3$.
\end{proof}
\begin{tav}
به طور مشابه می‌توان نشان داد که اگر 
$0<a<b$
آنگاه
\[
\lim_{n \to \infty}\sqrt[n]{a^n+b^n}=b.
\]
\end{tav}
\begin{mesal}
فرض کنید 
$\lim_{n \to \infty}a_n=3$
ثابت کنید که 
\[
\lim_{n \to \infty}\sqrt[n]{a_n}=1
\]
\end{mesal}
\begin{proof}[پاسخ]
از آنجا که 
$a_n \mapsto 3$
برای 
$\epsilon = \frac{1}{2}$
یک 
$N_\epsilon$
 موجود است، به طوری که
\[
\forall n >N_\epsilon \quad |a_n-3|<\frac{1}{2}
\]
یعنی
\[
\forall n>N_\epsilon \quad 2.5<a_n<3.5
\]
پس
\[
\forall n>N_\epsilon \quad \sqrt[n]{2.5}<\sqrt[n]{a_n}<\sqrt[n]{3.5}
\]
\[
\lim_{n \to \infty}\sqrt[n]{2.5}=1 , \lim_{n \to \infty}\sqrt[n]{2.5}=1
\]
بنا به قضیه‌ی فشردگی 
$\lim_{n \to \infty}\sqrt[n]{a_n}=1$
\end{proof}
\begin{tav}
\begin{enumerate}\hfill
\item
به طور مشابه می‌توان نشان داد که اگر 
$\lim_{n \to \infty}a_n=a>0$
آنگاه 
\[
\lim_{n \to \infty}\sqrt[n]{a_n}=1
\]
توجه کنید که شاید
$\epsilon=\frac{1}{2}$
در این جا کار نکند ولی می‌توان با انتخاب مناسبتری از آن به نتیجه‌ی مطلوب رسید. 
\item
در طی پاسخ مثال قبل همچنین ثابت کردیم که هر دنباله‌ی همگرا، کراندار است.
\end{enumerate}
\end{tav}
\begin{mesal}
حد دنباله‌ی زیر را بیابید:
\[
\sqrt[n]{2^n-1}
\]
\begin{proof}[راهنمائی]
داریم
\[
\sqrt[n]{2^n-1}=\sqrt[n]{1+2+2^2+\ldots+2^{n-1}}
\]
حال با استفاده از لم فشردگی نشان دهید که حد این دنباله برابر با ۲ است. 
\end{proof}
\end{mesal}
\begin{framed}
در این جلسه ثابت کردیم:
\begin{enumerate}
\item 
$(1+\frac{1}{n})^n$
همگراست.
\item 
اگر
$a>1$
آنگاه
$\lim_{n\to \infty} \sqrt[n]{a}=1$.
\item 
اگر 
$0<a<b$
آنگاه
\[
\lim_{n \to \infty}\sqrt[n]{a^n+b^n}=b.
\]
\item 
 اگر 
$\lim_{n \to \infty}a_n=a>0$
آنگاه 
\[
\lim_{n \to \infty}\sqrt[n]{a_n}=1
\]
\end{enumerate}

\end{framed}
\end{document}